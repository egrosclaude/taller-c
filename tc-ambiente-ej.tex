\section{Ejercicios}
\label{sec:ambienteej}
\begin{enumerate}
\item Escribir un programa que imprima una secuencia de números consecutivos donde
el valor inicial, el valor final y el incremento son dados como argumentos.
\item Mismo ejercicio pero donde los parámetros son pasados como variables de
ambiente.
\item Mismo ejercicio pero donde los parámetros son pasados como opciones.
\item Programar una calculadora capaz de resolver cálculos simples como los
siguientes:
\begin{lstlisting}
$ casio 3 + 5
8
$ casio 20 * 6
120
$ casio 5 / 3
1
\end{lstlisting}
\item Agregar la capacidad de fijar precisión (cantidad de decimales) como una
opción:
\begin{lstlisting}
$ casio -d2  5 / 3
1.66
\end{lstlisting}
\item Manteniendo la capacidad anterior, agregar la posibilidad de leer una
variable de ambiente que establezca la precisión default. Si no se da la
precisión como opción, se tomará la establecida por la variable de ambiente,
pero si se la especifica, ésta será la adoptada. Si no hay definida una
precisión se tomará 0. Ejemplo:
\begin{lstlisting}
$ casio 10 / 7
1
$ PRECISION_CASIO=5
$ export PRECISION_CASIO
$ casio 10 / 7
1.42857
$ casio -d2 10 / 7
1.42
\end{lstlisting}
\item Retomar ejercicios de programación de prácticas anteriores, agregándoles opciones. Por ejemplo, el programa para eliminar líneas de un archivo (Ejercicios \ref{sec:tc-esstandard-ej}) admite una opción para elegir líneas conteniendo o no conteniendo una cadena. El programa que cuenta palabras de un archivo (misma práctica) puede recibir opciones o variables de ambiente especificando cuáles serán los separadores entre palabras.
\end{enumerate}
