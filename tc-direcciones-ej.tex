\section{Ejercicios}
\label{sec:tc-direcciones-ej}
\begin{enumerate}
	\item Dado el programa siguiente, ¿a dónde apunta \texttt{k1}?
\begin{lstlisting}
main()
{
    int k;
    int *k1;
}
\end{lstlisting}
	\item Dado el programa siguiente, ¿a dónde apunta \texttt{m1}?
\begin{lstlisting}
int *m1;
main()
{
    ...
}	
\end{lstlisting}
	\item ¿Cuánto espacio de almacenamiento ocupa un arreglo de diez enteros? ¿Cuánto
espacio de almacenamiento ocupa un puntero a entero?
	\item Declarar variables \texttt{long} llamadas a, b y c, y punteros a \texttt{long} llamados p, q y r; y dar
las sentencias en C para realizar las operaciones siguientes. Para cada caso,
esquematizar el estado final de la memoria.
	\begin{enumerate}[label=\alph*.]
		\item Cargar p con la dirección de a. Si ahora escribimos \lstinline{*p = 1000}, ¿qué
      ocurre?
		\item Cargar r con el contenido de p. Si ahora escribimos \lstinline{*r = 1000}, ¿qué
      ocurre?
		\item Cargar q con la dirección de b, y usar q para almacenar una constante \texttt{4L}
      en el espacio de b.
		\item Cargar en c la suma de a y b, pero sin escribir la expresión $a+b$.
		\item Almacenar en c la suma de a y b pero haciendo todos los accesos a las
      variables en forma indirecta.
	\end{enumerate}
\item Compilar y ejecutar:
\begin{enumerate}[label=\alph*.]
		\item 
\begin{lstlisting}
main()
{
    char *a = "Ejemplo";
    printf("%s\n",a);
}
\end{lstlisting}

\item 
\begin{lstlisting}
main()
{
    char *a;
    printf("%s\n",a);
}
\end{lstlisting}		

\item 
\begin{lstlisting}
main()
{
    char *a = "Ejemplo";
    char *p;
    p = a;
    printf("%s\n", p);
}
\end{lstlisting}	
\end{enumerate}



	\item ¿Qué imprimirán estas sentencias?
	\begin{enumerate}[label=\alph*.]
\item \lstinline{char *s = "ABCDEFG";}
\item \lstinline{printf("%s\n", s);}
\item \lstinline{printf("%s\n", s + 0);}
\item \lstinline{printf("%s\n", s + 1);}
\item \lstinline{printf("%s\n", s + 6);}
\item \lstinline{printf("%s\n", s + 7);}
\item \lstinline{printf("%s\n", s + 8);}
\end{enumerate}
	\item ¿Son correctas estas sentencias? Bosqueje un diagrama del estado final de la
memoria para aquellas que lo sean.
	\begin{enumerate}[label=\alph*.]
\item \lstinline{char *a = "Uno";}
\item \lstinline{char *a, b; a = "Uno"; b = "Dos";}
\item \lstinline{char *a,*b ; a = "Uno"; b = a;}
\item \lstinline{char *a,*b ; a = "Uno"; b = *a;}
\item \lstinline{char *a,*b ; a = "Uno"; *b = a;}
\item \lstinline{char *a = "Dos"; *a = 'T';}
\item \lstinline{char *a = "Dos"; a = "T";}
\item \lstinline{char *a = "Dos"; *(a + 1) = 'i'; *(a + 2) = 'a';}
\item \lstinline{char *a, *b ; b = a;}
	\end{enumerate}
	\item Escribir funciones para:
	\begin{enumerate}[label=\alph*.]
\item Calcular la longitud de una cadena.
\item Dado un carácter determinado y una cadena, devolver la primera posición
      de la cadena en la que se lo encuentre, o bien $-1$ si no se halla.
\item Buscar una subcadena en otra, devolviendo un puntero a la posición donde
      se la halle.
	\end{enumerate}

	\item Escribir una función para reemplazar en una cadena todas las ocurrencias de
un carácter dado por otro, suponiendo:
	\begin{enumerate}[label=\alph*.]
\item Que no interesa conservar la cadena original, sino que se reemplazarán
      los caracteres sobre la misma cadena.
\item Que se pretende obtener una segunda copia, modificada, de la cadena
      original, sin destruirla.
	\end{enumerate}

	\item Escribir funciones para:
	\begin{enumerate}[label=\alph*.]
\item Rellenar una cadena con un carácter dado, hasta que se encuentre el $0$
      final, o hasta alcanzar $n$ iteraciones.
\item Pasar una cadena a mayúsculas o minúsculas.
	\end{enumerate}

	\item Reescriba dos de las funciones escritas en 8 y dos de las escritas en 10
usando la notación opuesta (cambiando punteros por arreglos).

\end{enumerate}
%TODO Ejercicios_Adicionales Ejercicios_Avanzados
