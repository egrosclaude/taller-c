
\section{Ejercicios}
\label{tc-constantes-ej}
\begin{enumerate}
	\item Indicar si las siguientes constantes están bien formadas, y en caso afirmativo indicar su tipo y dar su
valor decimal.
		\begin{multicols}{4}
		\begin{enumerate}[label=\alph*.]
\item \lstinline{'C'} 
\item \lstinline{'0'}
\item \lstinline{'0xAB'}
\item \lstinline{70}
\item \lstinline{1A}
\item \lstinline{0xABL}
\item \lstinline{070} 
\item \lstinline{'010'} 
\item \lstinline{0xaB}
\item \lstinline{080}
\item \lstinline{0x10} 
\item \lstinline{'0xAB'}
\item \lstinline{0XFUL}
\item \lstinline{'\030'} 
\item \lstinline{-40L}
\item \lstinline{015L}
\item \lstinline{x41}
\item \lstinline{'B'}
\item \lstinline{'\xBB'} 
\item \lstinline{'AB'}
\item \lstinline{322U}
	\end{enumerate}
	\end{multicols}
\item Indicar qué caracteres componen las constantes string siguientes:
		\begin{enumerate}[label=\alph*.]
\item \lstinline{"ABC\bU\tZ"}
\item \lstinline{"\103B\x41"}
	\end{enumerate}
\item ¿Cómo se imprimirán estas constantes string?
		\begin{enumerate}[label=\alph*.]
\item \lstinline{"\0BA"}
\item \lstinline{"\\0BA"}
\item \lstinline{"BA\0CD"}
	\end{enumerate}
\item ¿Qué imprime esta sentencia? Pista: \textit{nada} no es la respuesta correcta.
\begin{lstlisting}
printf("0\r1\r2\r3\r4\r5\r6\r7\r8\r9\r");
\end{lstlisting}
\item Escribir una macro que devuelva el valor numérico de un carácter correspondiente a un dígito en base decimal.
\item Idem donde el carácter es un dígito hexadecimal entre A y F.
\end{enumerate}

