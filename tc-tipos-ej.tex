
\section{Ejercicios}
\label{sec:tc-tipos-ej}
\begin{enumerate}
	\item ¿Cuáles de entre estas declaraciones contienen errores?
		\begin{multicols}{2}
		\begin{enumerate}[label=\alph*.]
			\item integer a;
			\item short i,j,k;
			\item long float (h);
			\item double long d3;
			\item unsigned float n;
			\item char 2j;
			\item int MY;
			\item float ancho, alto, long;
			\item bool i;
		\end{enumerate}
		\end{multicols}
	\item Dé declaraciones de variables con tipos de datos adecuados para almacenar:
	\begin{enumerate}[label=\alph*.]
		\item La edad de una persona.
		\item Un número de DNI.
		\item La distancia, en Km, entre dos puntos cualesquiera del globo terrestre.
		\item El precio de un artículo doméstico.
		\item El valor de la constante PI expresada con 20 decimales.
	\end{enumerate}
	\item Prepare un programa con variables conteniendo los valores máximos de cada tipo entero, para
comprobar el resultado de incrementarlas en una unidad. Imprima los valores de cada variable antes y
después del incremento. Incluya \textbf{unsigneds}.
	\item Lo mismo, pero dando a las variables los valores mínimos posibles, e imprimiéndolas antes y
después de decrementarlas en una unidad.
	\item Averigüe los tamaños de todos los tipos básicos en su sistema aplicando el operador \lstinline{sizeof()}.
	\item Si se asigna la expresión $(3-5)$ a un \lstinline{unsigned short}, ¿cuál es el resultado? ¿Depende de qué formato de conversión utilicemos para imprimirlo?
	\item ¿Qué hace falta corregir para que la variable \lstinline{r} contenga la división exacta de \lstinline{a} y \lstinline{b}?
	\begin{lstlisting}
int a, b;
float r;
a = 5;
b = 2;
r = a / b;		
	\end{lstlisting}
	\item ¿Qué resultado puede esperarse del siguiente fragmento de código?
	\begin{lstlisting}
int a, b, c, d;
a = 1;
b = 2;
c = a / b;
d = a / c;
	\end{lstlisting}
	\item ¿Cuál es el resultado del siguiente fragmento de código? Anticipe su respuesta en base a lo dicho en
esta unidad y luego confírmela mediante un programa.
	\begin{lstlisting}
printf("%d\n", 20/3);
printf("%f\n", 20/3);
printf("%f\n", 20/3.);
printf("%d\n", 10%3);
printf("%d\n", 3.1416);
printf("%f\n", (double)20/3);
printf("%f\n", (int)3.1416);
printf("%d\n", (int)3.1416);
	\end{lstlisting}
\item Escribir un programa que multiplique e imprima $100000 * 100000$. ¿De qué tamaño son los ints
en su sistema?
\item Convertir una moneda a otra sabiendo el valor de cambio. Dar el valor a dos decimales.
\item Escriba y corra un programa que permita saber si los chars en su sistema son signed o unsigned.
\item Escriba y corra un programa que asigne el valor 255 a un char, a un unsigned char y a un signed
char, y muestre los valores almacenados. Repita la experiencia con el valor \lstinline$-1$ y luego con \lstinline$'\377'$.
Explicar el resultado.
\item Copiar y compilar el siguiente programa. Explicar el resultado.
	\begin{lstlisting}
main() {
	double x;
	int i;
	i = 1400;
	x = i; /* conversion de int a double */
	printf("x = %10.6le\ti = %d\n",x,i);
	x = 14.999;
	i = x; /* conversion de double a int */
	printf("x = %10.6le\ti = %d\n",x,i);
	x = 1.0e+60;
	i = x;
	printf("x = %10.6le\ti = %d\n",x,i);
}
	\end{lstlisting}

\item Escriba un programa que analice la variable \lstinline{v} conteniendo el valor 347 y produzca la salida:
	\begin{lstlisting}
3 centenas
4 decenas
7 unidades
\end{lstlisting}
(y, por supuesto, salidas acordes si \lstinline{v} toma otros valores).
	\item Sumando los dígitos de un entero escrito en notación decimal se puede averiguar si es divisible por
3 (se constata si la suma de los dígitos lo es). ¿Esto vale para números escritos en otras bases? ¿Cómo
se puede averiguar esto?
	\item Indicar el resultado final de los siguientes cálculos
\begin{enumerate}[label=\alph*.]
\item \lstinline{int a; float b = 12.2; a = b;}
\item \lstinline{int a, b; a = 9; b = 2; a /= b;}
\item \lstinline{long a, b; a = 9; b = 2; a /= b;}
\item \lstinline{float a; int b, c; b = 9; c = 2; a = b/c;}
\item \lstinline{float a; int b, c; b = 9; c = 2; a = (float)(b/c);}
\item \lstinline{float a; int b, c; b = 9; c = 2; a = (float)b/c;}
\item \lstinline{short a, b, c; b = -2; c = 3; a = b * c;}
\item \lstinline{short a, b, c; b = -2; c = 3; a = (unsigned)b * c;}
\end{enumerate}
\item Aplicar operador cast donde sea necesario para obtener resultados apropiados:
	\begin{enumerate}[label=\alph*.] 
	\item 
	\begin{lstlisting}
int a; long b; float c;
a = 1; b = 2; c = a / b;
	\end{lstlisting} 

	\item 
	\begin{lstlisting}
long a; 
int b,c;
b = 1000; c = 1000;
a = b * c;
	\end{lstlisting}
	\end{enumerate}
\item ¿Cuál es la salida de este programa? ¿Cuál es la explicación?
\begin{lstlisting}
#include <stdio.h>
int main(){
	printf("%f\n", (float)333334126.98);
	printf("%f\n", (float)333334125.31);
}
\end{lstlisting}
\end{enumerate}
