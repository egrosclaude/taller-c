

\begin{preguntas}
\label{sec:tc-punteros-preg}
\question La expresión \code{&p} puede traducirse como
\choice \quotes{lo apuntado por \code{p}}.
\correctchoice \quotes{la dirección de \code{p}}.
\choice \quotes{el arreglo comenzado por \code{p}}.
\choice \code{*p}
\choice \quotes{la cadena apuntada por \code{p}}.

\question La expresión \code{a = *p} es equivalente a
\choice multiplicar \code{a} por \code{p}.
\correctchoice hacer \code{a} igual a lo apuntado por \code{p}.
\choice almacenar en \code{a} la dirección de \code{p}.
\choice \code{a = p}.

\question La expresión \code{&(*p)} es equivalente a
\choice la dirección de \code{p}.
\choice lo apuntado por \code{p}.
\choice lo apuntado por la dirección de \code{p}.
\correctchoice la dirección de lo apuntado por \code{p}.

\question La expresión \code{&(*p)} equivale al valor de 
\choice \code{*p}.
\choice \code{&p}.
\correctchoice \code{p}.
\choice \code{p * p}.

\question La expresión \code{*(&p)} equivale al valor de 
\choice \code{*p}
\choice \code{&p}
\correctchoice \code{p}
\choice \code{p * p}.

\question Con la declaración \code{char *k;}, la variable \code{k} será
\choice un carácter.
\correctchoice un puntero a carácter.
\choice la dirección de un carácter.
\choice todas las anteriores.

\question Con la declaración \code{char k;}, la variable \code{k} será
\correctchoice un carácter. 
\choice un puntero a carácter.
\choice la dirección de un carácter.
\choice todas las anteriores.

\question Con la declaración \code{int j;}, la expresión \code{&j} será
\choice un puntero.
\correctchoice equivalente a la dirección de \code{j}.
\choice equivalente a lo apuntado por \code{j}.
\choice un arreglo.

\question Con la declaración \code{int *j;}, la expresión \code{*j} será
\choice un puntero.
\choice equivalente a la dirección de j.
\correctchoice equivalente a lo apuntado por j.
\choice un arreglo.

\question La expresión \code{p-q}, si \code{p} y \code{q} son punteros a \code{char}, vale
\choice la cantidad de bytes entre las direcciones apuntadas por \code{p} y \code{q}.
\choice la diferencia entre las direcciones apuntadas por \code{p} y \code{q}.
\choice la cantidad de bytes que hace falta desplazarse desde la dirección apuntada por \code{p} para llegar a la dirección apuntada por \code{q}.
\correctchoice Todas las anteriores.
\choice Ninguna de las anteriores.

\question La expresión \code{g-f}, si \code{g} y \code{f} son punteros a \code{long}, vale
\choice La cantidad de bytes entre las direcciones apuntadas por \code{g} y \code{f}.
\correctchoice La cantidad de longs que caben entre las direcciones apuntadas por \code{g} y \code{f}.
\choice La diferencia entre los \code{long}s apuntados por \code{g} y por \code{f}.
\choice Todas las anteriores.
\choice Ninguna de las anteriores.

\question Con la declaración \code{char *s = "abcdef";} construimos
\choice un arreglo.
\choice un puntero a una cadena terminada en \lstinline{"0"}.
\correctchoice un puntero a una cadena terminada en \code{'\\0'}.
\choice un puntero a un carácter \code{'\\0'}.
\choice un puntero nulo.

\question El puntero nulo es igual a
\choice \lstinline{"0"}.
\correctchoice \code{(char *)0}.
\choice \code{(char) "0"}.
\choice \code{'0'}.

\question Con la declaración \code{char *a[] = \{"alfa", "beta", "gamma" \};}, se tiene que \code{a[1]} equivale a:
\correctchoice la dirección de la cadena \lstinline{"beta"}.
\choice la dirección de la cadena \lstinline{"alfa"}.
\choice la letra 'l' dentro de la cadena \lstinline{"alfa"}.
\choice la letra 'b' dentro de la cadena \lstinline{"beta"}.
\choice Ninguna de las anteriores.

\question Con la declaración \code{char *a[] = \{"alfa", "beta", "gamma" \};} se tiene que \code{*a[1]} equivale a:
\choice la dirección de la cadena \lstinline{"beta"}.
\choice la dirección de la cadena \lstinline{"alfa"}.
\choice la letra 'l' dentro de la cadena \lstinline{"alfa"}.
\correctchoice la letra 'b' dentro de la cadena \lstinline{"beta"}.
\choice Ninguna de las anteriores.

\question Con la declaración \code{char *a[] = \{"alfa", "beta", "gamma" \};}, la letra \code{'t'} dentro de la cadena \lstinline{"beta"} se puede escribir como:
\choice \code{a[1][2]}.
\choice \code{*(a[1]+2)}.
\choice \code{*(*(a+1)+2)}.
\correctchoice Todas las anteriores.
\choice Ninguna de las anteriores.

\end{preguntas}
