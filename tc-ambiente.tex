



\chapter{Comunicación con el ambiente}
\label{tc-ambiente}



Entendemos por comunicación con el ambiente todas aquellas formas posibles de
\textbf{intercambiar datos} entre el programa y el \textbf{entorno}, ya sea el sistema operativo,
el \textit{shell} del usuario, u otro programa que lo haya lanzado. Una necesidad
evidente de comunicación será recibir parámetros, argumentos u opciones de
trabajo. Otras necesidades serán generar archivos con resultados, o comunicar
una condición de error a la entidad que puso en marcha el programa.

\section{Redirección y piping}
Esta forma de comunicación en realidad no es específica del C sino que está
implementada (hoy, en prácticamente todos los sistemas operativos) por el shell
de usuario. Todos los programas en ejecución (o procesos) nacen con tres
canales de comunicación abiertos: \textbf{entrada standard, salida standard y salida
standard de errores}. Cuando el shell lanza un programa, por default le conecta
estos tres canales con los dispositivos lógicos teclado, pantalla y pantalla
respectivamente. El resultado es que el programa puede recibir caracteres por
teclado e imprimir cadenas por pantalla haciendo uso de las funciones de
entrada/salida corrientes.

Ahora bien, si el usuario indica al shell, en el momento de lanzar el programa,
que desea reconectar alguno de estos canales con otros dispositivos lógicos o
archivos, tenemos un fenómeno de \textbf{redirección}, que permite que el programa, sin
cambio alguno, utilice las mismas funciones de entrada/salida para leer y
generar archivos o comunicarse con dispositivos diferentes.

            Los procesos reciben tres canales de comunicación abiertos por donde
 relacionarse con el ambiente. Mediante redirección se pueden crear archivos con
           el producto de su salida, o alimentarlos con el contenido de archivos
                                                                  preexistentes.

Otra alternativa es el \textbf{piping}, o entubamiento, que permite, con un solo comando
de shell, el lanzamiento (en forma concurrente, si lo soporta el sistema
operativo) de dos o más procesos con sus entradas y salidas interconectadas
para funcionar acopladas. El shell se apoya en el sistema operativo para
construir un \textbf{pipe}, o tubería temporaria, para conducir el flujo de datos entre
los procesos que se comunican.

El C adhiere a las convenciones de redirección y piping permitiendo manejar
separadamente estos canales con sus funciones de Biblioteca Standard. Poder
realizar piping entre procesos permite separar arquitecturalmente las funciones
de un programa muy complejo, facilitando el desarrollo, aumentando la
mantenibilidad y fomentando la reutilización de los programas escritos sin
costo adicional de diseño o programación.

 Los procesos pueden comunicarse a través de pipes o tuberías. El sistema
 operativo UNIX hace uso extensivo de esta capacidad proveyendo una gran
 cantidad de comandos sencillos que, combinados mediante piping, permiten crear
 poderosas herramientas sin necesidad de programación.

Para poder aprovechar estas capacidades solamente se requiere un protocolo
común entre los programas que se comunicarán. Un medio para lograrlo, en
aquellos programas que no son naturalmente cooperativos, es a veces construir
adaptadores a nivel de shell. Estos son scripts generalmente sencillos que
transforman un formato de datos en otro, facilitando la flexibilidad que no da
el C por tratarse de un lenguaje compilado.

Los scripts, siendo interpretados, pueden ejecutarse directamente sin
compilación. Pueden modificarse y probarse más rápidamente que los programas
compilables, y la programación suele ser más flexible y poderosa. El costo
asociado con el scripting es una menor velocidad de ejecución, lo que propone
un estudio de cada caso, para optar entre scripting o programación ad hoc.
Ambientes como el moderno UNIX ofrecen numerosas herramientas y varios
intérpretes de lenguajes de scripting, cada cual con mayores ventajas en un
área determinada. Herramientas que es útil conocer son \code{grep}, \code{sed}, \code{diff}, \code{comm},
etc. El shell de usuario es normalmente una buena elección para scripting de
tareas simples, poseyendo un lenguaje completo con manejo de variables,
estructuras de control, arreglos, etc. Sin embargo, otros como \textbf{awk}, \textbf{Perl} o
\textbf{Python} tienen mejores capacidades de manejo de cadenas, esencial para el
trabajo que describimos, además de una sintaxis sumamente sintética y poderosa.

\section{Variables de ambiente}
El shell, responsable de recibir las órdenes del usuario para lanzar nuevos
procesos, mantiene áreas de memoria reservadas para \textbf{variables de ambiente} que
son accesibles a los nuevos procesos. Estas variables son simplemente pares
\textit{(nombre, valor)} de cadenas asociadas. 

Las variables de ambiente se pueden
establecer y consultar con comandos de shell, desde la línea de comandos o
desde un script; y lo mismo con funciones de Biblioteca Standard C desde un
programa compilado. Los programas pueden consultar una variable de ambiente y
decidir el curso de ejecución en función de su contenido; y pueden establecer
sus valores para los procesos hijos que originen. 

Las variables de ambiente son
una forma flexible de configurar el comportamiento de los programas.
Las funciones de manejo de variables de ambiente son \code{putenv()} y \code{getenv()} (POSIX).
Ver también \code{setenv()} y \code{unsetenv()} (BSD 4.3).

\begin{ejemplo}
Estos comandos a nivel de shell colocan una variable y su valor en el ambiente.
El comando \code{export} la hace visible a los procesos hijos.

\begin{lstlisting}
$ DIR=/usr/local/programa
$ export DIR
\end{lstlisting}

Para leer la variable desde un programa C:
\begin{lstlisting}
char directorio[50];
strcpy(directorio, getenv("DIR"));
\end{lstlisting}
\end{ejemplo}

\section{Argumentos de ejecución}
Un programa puede recibir argumentos al momento de ejecución, dados en la línea
de comandos. El protocolo para recibir argumentos se ha diseñado para ser lo
más general posible. Cada argumento en la línea de comandos es una cadena,
independientemente del tipo de los datos, y se accede desde el programa como un
puntero a carácter. Es responsabilidad del programa hacer las conversiones a
los tipos esperados.

Los argumentos son recibidos por la función \code{main()}, con las siguientes convenciones:
\begin{itemize}
	\item \code{main()} espera dos parámetros, un entero y un arreglo de punteros a
      carácter.
	\item El primer parámetro representa la cantidad total de argumentos en la
      línea de comandos, incluido el nombre del programa.
	\item Los elementos del segundo parámetro son punteros a cadenas, terminadas en
      \code{'\\0'}, representando cada argumento recibido (incluyendo el nombre del
      programa).
\end{itemize}
      
\begin{ejemplo}
El primer parámetro representa la cantidad total de argumentos en la
      línea de comandos, incluido el nombre del programa.

\begin{lstlisting}
main(int argc, char *argv[])
{
    if(argc != 3)
        printf("Debe dar nombre y edad del usuario\n");
    else
        printf("Nombre: %s Edad: %d\n", argv[1], atoi(argv[2]));
}	
\end{lstlisting}

Este programa se invocaría como:
\begin{lstlisting}
$ programa Alicia 26
Nombre: Alicia Edad: 26
\end{lstlisting}
\end{ejemplo}

\section{Salida del programa}
Cada programa ha sido lanzado por algún otro, por lo común el shell del
usuario. El programa puede seguir diferentes caminos de ejecución, encontrar
errores, condiciones en las cuales es imposible proseguir, etc. Al momento de
finalización del programa, puede ser interesante que el programa que le dio
origen recoja alguna indicación de este estado final. El C tiene la capacidad
(porque la tiene el sistema operativo) de devolver un entero, cuyo significado
queda completamente librado al programador. El programa originador debe
interpretar este código de retorno, que es una convención entre ambos
programas. Es costumbre, aunque para nada obligatoria, devolver un 0 en caso de
terminación exitosa, y números diferentes de cero para diferentes casos de
terminación con error, al estilo de los protocolos de las funciones de
Biblioteca Standard.

Esta característica es especialmente útil en el contexto de un script donde
necesitamos determinar si se debe proseguir la ejecución en base al estado
retornado por un programa invocado.
La función para terminar el programa devolviendo una señal de estado es \code{exit()}.
Si no se dan argumentos, el valor devuelto queda indefinido.


\begin{ejemplo}
\begin{lstlisting}
main(int argc, char *argv[])
{
    if(argc < 3) {
        printf("Insuficientes argumentos\n");
        exit(1);
    }
    procesar(argv[1],argv[2]);
    exit(0);
}
\end{lstlisting}
\end{ejemplo}

\section{Opciones}
Es muy común encontrar comandos del sistema operativo que aceptan un conjunto,
a veces muy vasto, de opciones. Las opciones, si están presentes, se reconocen
por comenzar con guiones, y deben ser los primeros argumentos dados al
programa.

La convención usual en UNIX de expresar las opciones con un signo guión y
letras, y opcionalmente argumentos numéricos, ha llevado a definir funciones de
Biblioteca Standard para manejar conjuntos de opciones.

\begin{ejemplo}
El siguiente fragmento de programa supone que tenemos escritas las funciones \code{transmitir()} y \code{recibir()}, y que con una opción dada al momento de ejecución queremos decidir cuál de ellas va a ser utilizada.
\begin{lstlisting}
#include <getopt.h>
#include <unistd.h>
extern char *optarg;
extern int optind, opterr, optopt;

int debug;

main(int argc, char *argv[])
{
    char *optstring="RrTtV:v:";
    int c;

    opterr=0;
    while((c=getopt(argc, argv, optstring)) != EOF)
        switch(c) {
            case 'v':
            case 'V':
                debug=atoi(optarg);
                printf("Nivel de debugging: %d\n",debug);
                break;
            case ':':
                printf("Falta valor numérico\n");
                exit(1);
                break;
            case 'R':
            case 'r':
                printf("Recibiendo\n");
                recibir(argv[optind]);
                break;
            case 'T':
            case 't':
                printf("Transmitiendo\n");
                transmitir(argv[optind]);
                break;
            case '?':
                printf("Mal argumento\n");
                break;
        }

}	
\end{lstlisting}

El programa podría usarse tanto para transmitir como para recibir archivos,
observando un nivel de salida de debugging conveniente. Podría invocarse como:

\begin{lstlisting}
$ transferir -v 2 -T archivo.txt
\end{lstlisting}
\end{ejemplo}

La función \code{getopt()} es quien va recogiendo las opciones vistas en la línea de
comandos y devolviéndolas como caracteres separados. La variable string
\code{optstring} contiene las opciones válidas. Para aquellas opciones (como \code{V} en el
ejemplo) que pueden asumir un modificador numérico, se ubica un símbolo \quotes{dos
puntos} a continuación en el string \code{optstring}. El valor para la opción numérica
se recibe en la variable \code{optarg}.

Si ocurre un error sintáctico en el procesamiento de las opciones, la rutina
devuelve el carácter \code{'?'} y emite un mensaje de error por salida de errores
standard. Si no se desea emitir este mensaje, se hace \code{opterr=0}.
Las funciones \code{recibir()} y \code{transmitir()} obtienen el nombre del archivo del
arreglo de argumentos \code{argv[]}, indexándolo con la variable \code{optind}, que queda
apuntando al siguiente elemento en la línea de comandos.

\section{Ejercicios}
\label{sec:ambienteej}
\begin{enumerate}
\item Escribir un programa que imprima una secuencia de números consecutivos donde
el valor inicial, el valor final y el incremento son dados como argumentos.
\item Mismo ejercicio pero donde los parámetros son pasados como variables de
ambiente.
\item Mismo ejercicio pero donde los parámetros son pasados como opciones.
\item Programar una calculadora capaz de resolver cálculos simples como los
siguientes:
\begin{lstlisting}
$ casio 3 + 5
8
$ casio 20 * 6
120
$ casio 5 / 3
1
\end{lstlisting}
\item Agregar la capacidad de fijar precisión (cantidad de decimales) como una
opción:
\begin{lstlisting}
$ casio -d2  5 / 3
1.66
\end{lstlisting}
\item Manteniendo la capacidad anterior, agregar la posibilidad de leer una
variable de ambiente que establezca la precisión default. Si no se da la
precisión como opción, se tomará la establecida por la variable de ambiente,
pero si se la especifica, ésta será la adoptada. Si no hay definida una
precisión se tomará 0. Ejemplo:
\begin{lstlisting}
$ casio 10 / 7
1
$ PRECISION_CASIO=5
$ export PRECISION_CASIO
$ casio 10 / 7
1.42857
$ casio -d2 10 / 7
1.42
\end{lstlisting}
\item Retomar ejercicios de programación de prácticas anteriores, agregándoles opciones. Por ejemplo, el programa para eliminar líneas de un archivo (Ejercicios \ref{sec:tc-esstandard-ej}) admite una opción para elegir líneas conteniendo o no conteniendo una cadena. El programa que cuenta palabras de un archivo (misma práctica) puede recibir opciones o variables de ambiente especificando cuáles serán los separadores entre palabras.
\end{enumerate}

