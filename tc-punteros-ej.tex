
\section{Ejercicios}
\label{sec:tc-punteros-ej}
\begin{enumerate}

\item ¿Qué objetos se declaran en las sentencias siguientes? El primero, por ejemplo, es un \textit{apuntador a función que recibe un entero y devuelve un double}.
\begin{enumerate}[label=\alph*.]
	\item double (*nu)(int kappa);
	\item int (*xi)(int *rho);
	\item long phi();
	\item int *chi;
	\item int pi[3];
	\item long *beta[3];
	\item int *(gamma[3]);
	\item int (*delta)[3];
	\item void (*eta[5])(int *rho);
	\item int *mu(long delta);
\end{enumerate}

\item Construir una función que reciba un arreglo de punteros a string A y un
string B, y busque a B en el array A, devolviendo su índice en el array, o bien
$-1$ si no se halla.
\item Construir una función que imprima una cadena en forma inversa.
Muestre una versión iterativa y una recursiva.
\item Construir un programa que lea una sucesión de palabras y las busque en un
pequeño diccionario. Al finalizar debe imprimir la cuenta de ocurrencias de
cada palabra en el diccionario.
\item Construir un programa que lea una sucesión de palabras y las almacene en un
arreglo de punteros a carácter.
\item Ordenar lexicográficamente el arreglo de punteros del ejercicio 4.
\end{enumerate}

%TODO Ejercicios_Adicionales Ejercicios_Avanzados
