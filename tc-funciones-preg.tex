

\begin{preguntas}
\label{sec:tc-funciones-preg}
\question ¿De qué tipo es la función siguiente?
\begin{lstlisting}
float z(int p, short q) {
	double g=1;
	float h=2;
	return g;
}
\end{lstlisting}
\choice \code{int}
\correctchoice \code{float}
\choice \code{short}
\choice \code{double}

\question ¿Qué ocurre con el parámetro \code{b} en el cuerpo de la función siguiente?
\begin{lstlisting}
int fun(int a) {
	a = 2 * b;
	return b;
}
\end{lstlisting} 
\choice El código no compila porque falta declarar el parámetro b.
\choice Se devuelve el valor de b que es basura por ser variable local.
\correctchoice Se devuelve el valor b siempre que b sea una global declarada más arriba.

\question ¿Cuál sería el prototipo más plausible para la función \code{q()} si su uso legal es como el siguiente?
\begin{lstlisting}
float p, r; 
int s;
r = q(p,s) / 2;	
\end{lstlisting}
\correctchoice \code{float q(float x, int y);}
\choice \code{float q(int x, int y);}
\choice \code{int q(float x, float y);}

\question ¿Cuál sería el prototipo más plausible para la función \code{t()} si su uso legal es como el siguiente?
\begin{lstlisting}
double w;
w = t(5e1, 2L);
\end{lstlisting}
\choice \code{long t();}
\choice \code{double t(int x, int y);}
\correctchoice \code{double t(double x, long y);}
\choice \code{long t(double x, double y);}

\question Con el prototipo:
\begin{lstlisting}
void fun1(long x, double y, int g, char h);
\end{lstlisting}
¿Cuál es el parámetro cuyo tipo \textbf{no es} correcto en la invocación de la función?
\begin{lstlisting}
fun1(500, 1.02e3, -12, 9);
\end{lstlisting}
\correctchoice x
\choice y
\choice g
\choice h

\question Con el prototipo:
\begin{lstlisting}
void fun2(char a, unsigned b, int c, double d);
\end{lstlisting}
¿Cuál es el parámetro cuyo tipo \textbf{no es} correcto en la invocación de la función?
\begin{lstlisting}
fun2('2', 100, 100, 100);
\end{lstlisting}
\choice a
\choice b
\choice c
\correctchoice d

\question ¿Con quién está relacionado el problema en estas líneas?
\begin{lstlisting}
void fun3(int e, unsigned short f, long int g, signed char h);
a = fun3(1, 1, 1, 1);
\end{lstlisting}
\choice Con e
\choice Con f
\choice Con g
\choice Con h
\correctchoice Con a

\end{preguntas}
