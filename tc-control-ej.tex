
\section{Ejercicios}
\label{tc-control-ej}

\begin{enumerate}
\item Reescribir estas sentencias usando while en vez de for:
\begin{enumerate}[label=\alph*.]
	\item 
\begin{lstlisting}
for(i=0; i<=10; i++)
    a = i;
\end{lstlisting}

\item 
\begin{lstlisting}
for( ; j<100; j+=2) {
    a = j;
    b = j * 2;
}
\end{lstlisting}

\item 
\begin{lstlisting}
for( ; ; )
    a++;
\end{lstlisting}
\end{enumerate}

\item Si la función \lstinline{quedanDatos()} devuelve el valor lógico que sugiere su nombre,
¿cuál es la estructura preferible?
\begin{enumerate}[label=\alph*.]
\item 
\begin{lstlisting}
while(quedanDatos()) {
    procesar();
}
\end{lstlisting}

\item 
\begin{lstlisting}
do {
    procesar();
} while(quedanDatos());
\end{lstlisting}
\end{enumerate}
\item ¿Cuál es el error de programación en estos ejemplos?
	\begin{enumerate}[label=\alph*.]
\item 
\begin{lstlisting}
for(i = 0; i < 10; i++);
    a = i - 50L;
\end{lstlisting}

\item 
\begin{lstlisting}
while(i < 100) {
    procesar(i);
    a = a + i;
}
\end{lstlisting}
\end{enumerate}
\item ¿Cuál es el valor de x a la salida de los lazos siguientes?
	\begin{enumerate}[label=\alph*.]
	\item 
\begin{lstlisting}
for(x = 0; x<100; x++);
\end{lstlisting}

	\item 
\begin{lstlisting}
for(x = 32; x<55; x += 3);
\end{lstlisting}

	\item 
\begin{lstlisting}
for(x =  10;x>0; x--);
\end{lstlisting}
\end{enumerate}
\item ¿Cuántas X escriben estas líneas?
\begin{lstlisting}
for (x = 0; x < 10; x++)
    for (y = 5;  y >  0; y--)
        escribir("X");
\end{lstlisting}
\item Escribir sentencias que impriman la tabla de multiplicar para un entero
dado.
\item Imprimir la tabla de los diez primeros números primos (sólo divisibles por
sí mismos y por la unidad).
\item Escribir las sentencias para calcular el factorial de un entero.
\end{enumerate}
%TODO Ejercicios_Adicionales Ejercicios_Avanzados

