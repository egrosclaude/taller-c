\section{Ejercicios}
\label{sec:tc-variables-ej}
\begin{enumerate}
	\item Copie, compile y ejecute el siguiente programa. Posteriormente agregue un modificador static sobre
la variable a y repita la experiencia.
\begin{lstlisting}
int fun()
{
	int a;
	a = a + 1;
	return a;
}
main()
{
	printf("%d\n", fun());
	printf("%d\n", fun());
}
\end{lstlisting}
\item ¿Qué imprime este programa?
\begin{lstlisting}
int alfa;
int fun()
{
	int alfa;
	alfa = 1;
	return alfa;
}
main()
{
	alfa = 2;
	printf("%d\n",fun());
	printf("%d\n",alfa);
}
\end{lstlisting}
\item ¿Qué imprime este programa?
\begin{lstlisting}
int alfa;
int fun(int alfa)
{
	alfa = 1;
	return alfa;
}
main()
{
	alfa = 2;
	printf("%d\n",fun(alfa));
	printf("%d\n",alfa);
}
\end{lstlisting}
 \item Copie y compile, juntas, las unidades de traducción que se indican abajo. ¿Qué hace falta para que
la compilación sea exitosa?

	
\begin{tabular}{p{4cm}|p{5cm}}
\hline
fuente1.c & fuente2.c\\
\hline
\begin{codecell}
int a;
int fun1(int x)
{
	return 2 * x;
}
\end{codecell}
&
\begin{codecell}
main()
{
	a = 1;
	printf("d\n", fun1(a));
}
\end{codecell}
\\
\end{tabular} 

\item ¿Qué ocurre si un fuente intenta modificar una variable externa, declarada en otra unidad de
traducción como \lstinline{const}? Prepare, compile y ejecute un ejemplo.
\item ¿Qué resultado puede esperarse de la compilación de estos fuentes?

\begin{tabular}{p{4cm}|p{5cm}|p{4cm}}
\hline
header.h & fuente1.c &fuente2.c\\
\hline
\begin{codecell}
#include <stdio.h>
#define VALOR 6
\end{codecell}
&
\begin{codecell}
#include "header.h"
main()
{
	static int c;
	printf("%d\n",fun(c));
}
\end{codecell}
&
\begin{codecell}
#include "header.h"
int fun(int x)
{
	return VALOR * x;
}
\end{codecell}
\\
\end{tabular}

\item Denotemos esquemáticamente que un módulo objeto \lstinline{prueba.o} contiene un objeto de datos
\lstinline{x} y una función \lstinline{fun()}, ambos de liga externa, de esta manera:


\begin{tabular}{|c|}
\hline
prueba.o\\
\hline
x\\
fun()\\
\hline
\end{tabular}

Si se tiene un conjunto de archivos y unidades de traducción que se compilarán para formar los
respectivos módulos objeto, ¿cómo se aplicaría la notación anterior al conjunto de módulos objeto
resultantes? Hacer el diagrama para los casos que aparecen en el Cuadro \ref{tab:fuentes}. ¿Hay colisión de nombres? ¿Hay referencias que el linker no pueda resolver? Cada grupo de fuentes, ¿puede producir un ejecutable?

\item  Un conjunto de programas debe modelar eventos relativos a un aeropuerto. Se necesita preparar una
implementación de las estructuras de datos y funciones del aeropuerto, para ser usada por los demás
programas. Especifique las variables y funciones (en pseudocódigo) que podrán satisfacer los
siguientes requerimientos. Preste atención a las declaraciones extern y static.
\begin{itemize}
	\item El aeropuerto tendrá cinco pistas.
\item Se mantendrá un contador de la cantidad total de aviones en el aeropuerto y uno de la cantidad
total de aviones en el aire.
\item Para cada pista se mantendrá la cantidad de aviones esperando permiso para despegar de ella y la
cantidad de aviones esperando permiso para aterrizar en ella.
\item Habrá una función para modelar el aterrizaje y otra para modelar el despegue por una pista dada
(decrementando o incrementando convenientemente la cantidad de aviones en una pista dada, en
tierra y en el aire).
\item Habrá una función para consultar, y otra para establecer, la cantidad de aviones esperando
aterrizar o despegar por cada pista.
\item Habrá una función para consultar la cantidad de aviones en tierra y otra para consultar la cantidad
de aviones en el aire.
\item No deberá ser posible que un programa modifique el estado de las estructuras de datos sino a
través de las funciones dichas.
\end{itemize}
\item ¿Cuáles pueden ser los \quotes{agentes externos} al programa que sean capaces de cambiar el valor de una variable \lstinline{volatile}?
\end{enumerate}


\begin{table}
\centering
\begin{tabular}{m{0.2cm}|m{3cm}|m{3.5cm}|m{3.5cm}|m{3cm}}
\hline
& hdr1.h & fuente1.c & fuente2.c & fuente3.c\\
\hline

a)
&
\begin{codecell}
#define UNO 1   
#define DOS 2 
extern int a;
\end{codecell}
&
\begin{codecell}
#include "hdr1.h"
main() 
{ 
	int b;
	b = fun1(a);
}
\end{codecell}
&
\begin{codecell}
#include "hdr1.h"
int fun1(int x) 
{
	return x+fun2(x);
}
static int 
	fun2(int x) 
{
	return x + DOS;
}
\end{codecell}
&
\begin{codecell}
#include "hdr1.h"
int a;
\end{codecell}\\
\hline



b)
&
\begin{codecell}
extern int c;
extern int 
	fun1(int p), 
	fun2(int p);
\end{codecell}
&
\begin{codecell}
#include "hdr1.h"
int fun1(int x)
{
	return 
		fun2(x)+1;
}
\end{codecell}
&
\begin{codecell}
int a, b, c=1;
int fun2(int x)
{
	return x-1;
}
\end{codecell}
&
\begin{codecell}
main()
{
	int d;
	d = fun1(3);
}
\end{codecell}\\
\hline



c)
&
\begin{codecell}
 
\end{codecell}
&
\begin{codecell}
int fun1(int x)
{
	return x+1;
}
int fun2(int x)
{
	return x+2;
}
\end{codecell}
&
\begin{codecell}
int fun3(int x)
{
	return x+3;
}
static int 
	fun2(int x)
{
	return x+4;
}
\end{codecell}
&
\begin{codecell}
main()
{
	int a;
	a = fun1(a);
}
\end{codecell}\\
\hline

d)
&
\begin{codecell}

\end{codecell}
&
\begin{codecell}
int fun1(int x)
{
	extern int b;
	x = b-fun2(x);
}
\end{codecell}
&
\begin{codecell}
static int 
	a = 1;
static int 
	b = 1;
int fun2(int x) 
{
	return x-a;
}
\end{codecell}
&
\begin{codecell}
int b;
main()
{
	b = 2;
	printf("%d",
		fun1(3));
}
\end{codecell}\\

\end{tabular}
\caption{Conjuntos de fuentes y propiedad de liga}
\label{tab:fuentes}
\end{table}

