
\section{Ejercicios}
\label{sec:bstandardej}
\begin{enumerate}
	\item Utilizar la función de cantidad variable de argumentos definida más arriba
para obtener los promedios de los 2, 3, ..., n primeros elementos de un
arreglo.
	\item Construir una función de lista variable de argumentos que efectúe la
concatenación de una cantidad arbitraria de cadenas en una zona de memoria
provista por la función que llama.
	\item Construir una función de cantidad variable de argumentos que sirva para
imprimir, con un formato especificado, mensajes de debugging, conteniendo
nombres y valores de variables.
	\item Construir un programa que separe la entrada standard en palabras, usando las
macros de clasificación de caracteres. Debe considerar como delimitadores a los
caracteres espacio, tabulador, signos de puntuación, etc.
	\item Dadas dos fechas y horas del día, calcular su diferencia. Utilizar las
funciones de Biblioteca Standard para convertir a tipos de datos convenientes e imprimir la
diferencia en años, meses, días, horas, etc.
	\item Generar fechas al azar dentro de un período de tiempo dado.
\end{enumerate}
