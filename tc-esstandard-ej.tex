
\section{Ejercicios}
\label{sec:tc-esstandard-ej}
\begin{enumerate}
\item Escribir una función que copie la entrada en la salida pero eliminando las
vocales.
\item Escribir una función que reemplace los caracteres no imprimibles por
caracteres punto.
\item Construir un programa que imprima su entrada invertida, carácter por carácter.
\item Construir un programa que imprima su entrada invertida, línea por línea.
\item Construir un programa que cuente la cantidad de palabras de un archivo,
separadas por blancos, tabuladores o fin de línea.
\item Construir un programa que cuente la cantidad de caracteres y de líneas de un
archivo.
\item Construir un programa que permita eliminar de un archivo las líneas que
contengan una cadena dada.
\item Escribir una función que reciba como argumento dos enteros y devuelva un
string de formato conteniendo una máscara de formato apropiada para imprimir un
número en punto flotante. Por ejemplo, si se le dan como argumentos 7 y 2,
deberá devolver el string \lstinline{"%7.2f"}. Aplicar la función para imprimir números en
punto flotante.
\item Escribir sobre un archivo una variable \code{int} con valor 1 y una variable \code{long}
con valor 2. Hacerlo primero con funciones de E/S con formato, y luego con
funciones de acceso directo. Examinar en cada caso el resultado visualizando el
archivo y opcionalmente con un comando como \code{od -bc}.
\item (*) Defina una estructura básica simple para un registro, a su gusto. Puede ser
un registro de información personal, bibliográfica, etc. Construya funciones
para leer estos datos del teclado e imprimirlos en pantalla. Luego, usando
funciones ANSI C, construya funciones para leer y escribir una de estas
estructuras en un archivo, dado un número de registro lógico determinado.
\item (*) Repita el ejercicio anterior reemplazando las funciones ANSI C por funciones
POSIX.
\item Construya programas que utilicen las funciones anteriores, ANSI C o POSIX,
y ofrezcan un menú de operaciones de administración: cargar un dato en el
archivo, imprimir los datos contenidos en una posición determinada, listar la
base generada completa, eliminar un registro, etc.
\item Construir una función que pida por teclado datos personales (nombre, edad, ...) y los almacene en una estructura. Construir una función que imprima los valores recogidos por la función anterior. 
\item Construir un programa que lea una cantidad de datos en lote y luego los imprima utilizando las funciones del ejercicio anterior. Generar un archivo de datos usando redirección.
\item Realizar el mismo programa del punto anterior pero efectuando toda la E/S sobre archivos. El programa deberá poder leer el archivo de datos del punto anterior.
\item Construir un programa que organice la salida de la función anterior para obtener un clon del comando \code{od} de UNIX.
\item Construir un programa que lea el listado de un directorio en formato largo (la salida del comando \code{ls -l}) y devuelva la cantidad total de bytes ocupados.
\item Construya programas que utilicen las funciones ANSI C o POSIX de los ejercicios anteriores \textbf{marcados con un asterisco}, y que ofrezcan un menú de operaciones de administración: cargar un dato en el archivo, imprimir los datos contenidos en una posición determinada, listar la base generada completa, eliminar un registro, etc.
\end{enumerate}

